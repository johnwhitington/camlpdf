\documentclass[a4paper]{memoir}
\usepackage{palatino}
\usepackage{microtype}
\usepackage{graphics}
\usepackage[plainpages=false,pdfpagelabels,pdfborder=0 0 0]{hyperref}
\usepackage{upquote}
\newcommand{\smallgap}{\vspace{4mm}}
\newcommand{\cpdf}{\texttt{cpdf}}
\addtolength{\textwidth}{20mm}
\makeindex
\begin{document}
\frontmatter
\thispagestyle{empty}

\begin{flushright}

{\sffamily \bfseries \Huge An Introduction to PDF with CamlPDF}

\vspace{4mm}{{\LARGE John Whitington}\\ \today}

\vfill

\includegraphics{logo.pdf}

\vspace{2mm}
{\sffamily \bfseries \LARGE Coherent Graphics Ltd}

\end{flushright}

\clearpage

\thispagestyle{empty}
\noindent For bug reports, feature requests and comments, email\\ \texttt{contact@coherentgraphics.co.uk}

\vspace*{\fill}
\noindent\copyright 2010-2022 Coherent Graphics Limited. All rights reserved.

\smallgap 
\noindent Adobe, Acrobat, Adobe PDF, Adobe Reader and PostScript are
registered trademarks of Adobe Systems Incorporated.

% Letter
\cleardoublepage
% \tableofcontents

\mainmatter

\chapterstyle{hangnum}
\pagestyle{ruled}
\section*{A First Program}
You will require \texttt{ocamlfind} to be installed on your system. It comes with any modern installation of OCaml.

To build \textsf{CamlPDF}, navigate to the source directory and type
\begin{framed}
\verb!make!
\end{framed}

\noindent You can then install \textsf{CamlPDF}:
\begin{framed}
\verb!make install!
\end{framed}


\noindent To build the examples, navigate to the examples folder and type
\begin{framed}
\verb!make!
\end{framed}
\noindent Now run a simple example to build the file \verb!hello.pdf!
\begin{framed}
\verb!./pdfhello!
\end{framed}

\noindent As an alternative to compiling CamlPDF yourself, you may use the \texttt{OPAM} package manager, if you have it installed:
\begin{framed}
\verb!opam install camlpdf!
\end{framed}

\noindent Then (or otherwise) we may load CamlPDF into any other top level:

\begin{framed}
\begin{verbatim}
ocaml
        Objective Caml
# #use "topfind";;
- : unit = ()
Findlib has been successfully loaded. Additional directives:
  #require "package";;      to load a package
  #list;;                   to list the available packages
  #camlp4o;;                to load camlp4 (standard syntax)
  #camlp4r;;                to load camlp4 (revised syntax)
  #predicates "p,q,...";;   to set these predicates
  Topfind.reset();;         to force that packages will be
reloaded
  #thread;;                 to enable threads

- : unit = ()
# #require "camlpdf";;
/Users/john/.opam/4.00.1/lib/ocaml/unix.cma: loaded
/Users/john/.opam/4.00.1/lib/ocaml/bigarray.cma: loaded
/Users/john/.opam/4.00.1/lib/camlpdf: added to search path
/Users/john/.opam/4.00.1/lib/camlpdf/camlpdf.cma: loaded

#
\end{verbatim}
\end{framed}
\noindent The \textsf{Pdfread} module allows us to load PDF files into memory. The raw PDF data is parsed into a structured OCaml value of type \textsf{Pdf.pdfdoc}:
\begin{framed}
\begin{verbatim}
# let pdf = Pdfread.pdf_of_file None None "hello.pdf";;
val pdf : Pdf.t =
  {Pdf.major = 1; Pdf.minor = 1; Pdf.root = 2;
   Pdf.objects =
    {Pdf.maxobjnum = 4; Pdf.parse = Some <fun>;
     Pdf.pdfobjects = <abstr>;
     Pdf.object_stream_ids = <abstr>};
   Pdf.trailerdict =
    Pdf.Dictionary
     [("/Root", Pdf.Indirect 2);
      ("/ID",
       Pdf.Array
        [Pdf.String "<elided>";
         Pdf.String "<elided>"]);
      ("/Size", Pdf.Integer 4)]}
\end{verbatim}
\end{framed}
\noindent Looking at the parts of the \textsf{Pdf.t} record type:
\begin{itemize}
\item \textsf{Pdf.major} and \textsf{Pdf.minor} - the parts of the PDF version number. Here, PDF Version 1.1
\item \textsf{Pdf.root} - the object number of the 'root object' of the PDF (A PDF is a directed graph of objects, indexed by number)
\item \textsf{Pdf.objects} - the PDF objects
\item \textsf{Pdf.trailerdict} - the trailer dictionary. This is a distinguished PDF object containing a number of commonly used per-file items. \textsf{Pdf.trailerdict} has type \textsf{Pdf.pdfobject}, which represents all possible PDF data.
\end{itemize}

\section*{Diversion: A Look at hello.pdf}
Here is the contents of the file \texttt{hello.pdf}, as you might see it in a text editor, annotated with some explanatory comments.
\begin{framed}
\noindent\small\verb!%PDF-1.1! \textit{Header}\\
\noindent\small\verb!%%$^@!\\
\noindent\small\verb!1 0 obj! \textit{Object 1\ldots}\\
\noindent\small\verb!<< /Type /Pages /Kids [ 3 0 R ] /Count 1 >>! \textit{\ldots which is the catalogue of pages}\\
\noindent\small\verb!endobj!\\
\noindent\small\verb!2 0 obj! \textit{This object is a stream, which is a dictionary plus some binary data}\\
\noindent\small\verb!<< /Length 102 >>!\\
\noindent\small\verb!stream! \textit{Usually compressed, but plain here for ease of reading}\\
\noindent\small\verb!1.000000 0.000000 0.000000 1.000000 50.000000 770.000000 cm!\\
\noindent\small\verb$BT /F0 36.000000 Tf (Hello, World!) Tj ET $\textit{The page content, a bit like PostScript}\\
\noindent\small\verb!endstream!\\
\noindent\small\verb!endobj!\\
\noindent\small\verb!3 0 obj! \textit{The page object, }\\
\noindent\small\verb!<< /Type /Page!\\
\noindent\small\verb!   /Parent 1 0 R! \textit{The syntax "1 0 R" means a reference to Object 1}\\ 
\noindent\small\verb!   /Resources!\\
\noindent\small\verb!      << /Font! \textit{The font dictionary}\\
\noindent\small\verb!            << /F0!\\
\noindent\small\verb!              << /Type /Font /Subtype /Type1 /BaseFont /Times-Italic >>!\\
\noindent\small\verb!            >> >>!\\
\noindent\small\verb!   /MediaBox [ 0.000000 0.000000 595.275591 841.889764 ]! \textit{The page dimensions}\\
\noindent\small\verb!   /Rotate 0!\\
\noindent\small\verb!   /Contents [ 2 0 R ] >>! \textit{Reference to contents in object 2}\\
\noindent\small\verb!endobj!\\
\noindent\small\verb!4 0 obj!\\
\noindent\small\verb!<< /Type /Catalog /Pages 1 0 R >>! \textit{The root object}\\
\noindent\small\verb!endobj!\\
\noindent\small\verb!xref! \textit{The cross-reference table, listing the byte offsets of each object for random access.}\\
\noindent\small\verb!0 5 !\\
\noindent\small\verb!0000000000 65535 f !\\
\noindent\small\verb!0000000015 00000 n !\\
\noindent\small\verb!0000000074 00000 n !\\
\noindent\small\verb!0000000227 00000 n !\\
\noindent\small\verb!0000000449 00000 n !\\
\noindent\small\verb!trailer! \textit{The trailer dictionary}\\
\noindent\small\verb!<< /Size 5 /Root 4 0 R /ID [ (<elided>) (<elided>) ] >>!\\
\noindent\small\verb!startxref!\\
\noindent\small\verb!498! \textit{The trailer}\\
\noindent\small\verb!%%EOF!\\
\end{framed}
\section*{Saving the Document}
The \textsf{Pdf.t} data type is a record of mutable values. Let's change the PDF Version number and write the file.
\begin{framed}
\begin{verbatim}
# pdf.Pdf.minor <- 2;;
- : unit = ()
# Pdfwrite.pdf_to_file pdf "hello2.pdf";;
- : unit = ()
\end{verbatim}
\end{framed}
\section*{Next Steps}
The objects in a PDF document are of type \textsf{Pdf.pdfobject}:
\begin{framed}
\noindent\verb!type stream =! \textit{Stream data. Either in memory or still in the file}\\
\verb!  | Got of Utility.bytestream!\\
\verb!  | ToGet of Pdfio.input * int * int! \textit{input, offset, length}\\
\verb!!\\
\verb!type pdfobject =!\\
\verb!  | Null!\\
\verb!  | Boolean of bool!\\
\verb!  | Integer of int!\\
\verb!  | Real of float!\\
\verb!  | String of string!\\
\verb!  | Name of string!\\
\verb!  | Array of pdfobject list!\\
\verb!  | Dictionary of (string * pdfobject) list!\\
\verb!  | Stream of (pdfobject * stream) ref! \textit{Stream data (see above)}\\
\verb!  | Indirect of int! \textit{A reference to another object}\\
\end{framed}

\noindent For instance the PDF object in the file:

\begin{framed}
\begin{verbatim}
3 0 obj
<< /Type /Page
   /Parent 1 0 R
   /MediaBox [ 0.000000 0.000000 595.275591 841.889764 ]
   /Rotate 0
   /Contents [ 2 0 R ]
>>
end
\end{verbatim}
\end{framed}

\noindent is represented as object number 3 with the Pdf.t instance:
\begin{framed}
\small\begin{verbatim}
Dictionary
  ["/Type", Name "/Page";
   "/Parent", Indirect 1;
   "/MediaBox",
       Array [Real 0.; Real 0.; Real 595.275591; Real 841.889764];
   "/Rotate", Integer 0;
   "/Contents", Array [Indirect 2]]
\end{verbatim}
\end{framed}

\section*{Working with Pages}
Introduce a command to show the current document, using whatever command opens (or updates) a PDF view on your system:
\begin{framed}
\noindent\verb!let show pdf =!\\
\verb!  Pdfwrite.pdf_to_file pdf "temp.pdf";!\\
\verb!  ignore (Sys.command "open temp.pdf");;! \textit{Customize here}\\
\end{framed}

\noindent The \textsf{Pdfpage} module deals with PDF pages. We can get the list of pages from a document:
\begin{framed}
\small\begin{verbatim}
# let pages = Pdfpage.pages_of_pagetree pdf;;
val pages : Pdfpage.t list =
  [{Pdfpage.content = [Pdf.Indirect 4];
    Pdfpage.mediabox =
     Pdf.Array
      [Pdf.Integer 0; Pdf.Integer 0; Pdf.Real 595.275590551;
       Pdf.Real 841.88976378];
    Pdfpage.resources =
     Pdf.Dictionary
      [("/Font",
        Pdf.Dictionary
         [("/F0",
           Pdf.Dictionary
            [("/Type", Pdf.Name "/Font");
             ("/Subtype", Pdf.Name "/Type1");
             ("/BaseFont", Pdf.Name "/Times-Italic")])])];
    Pdfpage.rotate =
      Pdfpage.Rotate0; Pdfpage.rest = Pdf.Dictionary []}]
\end{verbatim}
\end{framed}
\noindent Each page is a record containing five things:
\begin{itemize}
\item \textsf{Pdfpage.content} An ordered list of pdf objects representing the one or more streams containing the graphical data for the page.
\item \textsf{Pdfpage.mediabox} The page dimensions
\item \textsf{Pdfpage.resources} The resources dictionary for a page, which contains the fonts, colour spaces and so on for the page.
\item \textsf{Pdfpage.rotate} The viewing rotation for the page.
\item \textsf{Pdfpage.rest} The rest of the page dictionary (i.e that which has not been separated into the items above).
\end{itemize}
Let's change the viewing rotation to 90 degrees:
\begin{framed}
\small\begin{verbatim}
# let page = {(List.hd pages) with Pdfpage.rotate = Pdfpage.Rotate90};;
val page : Pdfpage.t = ...
# let pdf = Pdfpage.change_pages false pdf [page];;
val pdf : Pdf.t = ...
# show pdf;;
- : unit
\end{verbatim}
\end{framed}
\noindent Now change the rotation back: we're going to work with graphics next, and the viewing roation would confuse:
\begin{framed}
\small\begin{verbatim}
# let page = List.hd pages;;
val page : Pdfpage.t = ...
# let pdf = Pdfpage.change_pages false pdf [page];;
val pdf : Pdf.t = ...
# show pdf;;
- : unit
\end{verbatim}
\end{framed}
\section*{Graphics and Text}
The \textsf{Pdfops} module represents the graphical content of each page, which is formed of PostScript-like operators which draw the page. Let's get the operator list from the page:
\begin{framed}
\small\begin{verbatim}
# let ops =
    Pdfops.parse_operators
      pdf page.Pdfpage.resources page.Pdfpage.content;;
val ops : Pdfops.t list =
  [Pdfops.Op_cm
    {Transform.a = 1.; Transform.b = 0.;
     Transform.c = 0.; Transform.d = 1.;
     Transform.e = 50.; Transform.f = 770.};
   Pdfops.Op_BT;
   Pdfops.Op_Tf ("/F0", 36.);
   Pdfops.Op_Tj "Hello, World!";
   Pdfops.Op_ET]
\end{verbatim}
\end{framed}
\noindent The \textsf{Op\_cm} operator alters the graphics matrix to position the text. \textsf{Op\_BT} and \textsf{Op\_ET} mark the beginning and end of a text section. \textsf{Op\_Tf} chooses 36pt Times Italic (which is font \textsf{F0} in the page's font dictionary in its resources) and \textsf{Op\_Tj} paints the text.

Let's add operators to underline the text -- \textsf{Op\_m} to move, \textsf{Op\_l} to draw a line and \textsf{Op\_S} to stroke the path. We calculate the width of the underline using the \textsf{Pdfstandard14} and \texttt{Pdftext} modules to get the raw width of the string in millipoints, adjusting for font size and converting to points.

\begin{framed}
\small\begin{verbatim}
# let width =
    Pdfstandard14.textwidth false Pdftext.TimesItalic "Hello, World!";;
val width : int = 5555

# let actual_width = float width *. 36. /. 1000.;;
val actual_width : float = 199.98

# let ops' =
    ops @
      [Pdfops.Op_m (0., 0.);
       Pdfops.Op_l (actual_width, 0.);
       Pdfops.Op_S];;
val ops' : Pdfops.t list = ...
\end{verbatim}
\end{framed}

\noindent and make the new content stream:

\begin{framed}
\small\begin{verbatim}
# let stream = Pdfops.stream_of_ops ops';;
val stream : Pdf.pdfobject =
  Pdf.Stream
   {contents =
     (Pdf.Dictionary [("/Length", Pdf.Integer 72)], Pdf.Got <abstr>)}
\end{verbatim}
\end{framed}

\noindent and add it to the page, and replace the page in the PDF.

\begin{framed}
\small\begin{verbatim}
# let page' = {page with Pdfpage.content = [stream]};;
val page' : Pdfpage.t = ...

# let pdf = Pdfpage.change_pages false pdf [page'];;
val pdf : Pdf.t = ...
\end{verbatim}
\end{framed}

\noindent and show it:

\begin{framed}
\small\begin{verbatim}
# show pdf;;
- : unit ()
\end{verbatim}
\end{framed}

\section*{Next Steps}

\noindent CamlPDF is a large piece of software. A good way to get to know it is to study the examples shipped with \textsf{CamlPDF}:
\smallgap

{\centering\small
\begin{tabular}{ll}
\texttt{pdfhello.ml} & Build a "Hello, World!" PDF from scratch\\
\texttt{pdfdecomp.ml} & Command line utility to decompress a PDF\\
\texttt{pdfmerge.ml} & Command line utility to merge PDF files\\
\texttt{pdfdraft.ml} & Command line utility to make draft documents\\
\texttt{pdftest.ml} & Reads and interprets a file to test CamlPDF's major functionality\\
\texttt{pdfencrypt.ml} & Command line utility to encrypt a PDF file\\
\end{tabular}
}


\smallgap

\noindent Summary of \textsf{CamlPDF} modules:

\smallgap

{\centering\small
\begin{tabular}{lp{10cm}l}
  \textbf{Module} & \textbf{Description} \\
Pdfutil &
General Functions. & \\
Pdfio &
Generic Input/Ouput from/to channels, strings, files etc. & \\
Pdftransform &
Affine Transformations in Two Dimensions &  \\
Pdfunits &
Units and Unit Conversion & \\
Pdfpaper &
Media Sizes & \\
Pdf &
Representing PDF Files in Memory & \\
Pdfcrypt &
Decrypting PDF files & \\
Pdfflate &
Interface to miniz.c via Zlib-like functions. & \\
Pdfcodec &
Encoding and Decoding PDF Streams & \\
Pdfwrite &
Writing PDF Files & \\
Pdfgenlex &
A very fast lexer for very basic tokens. & \\
Pdfread &
Reading PDF Files & \\
Pdfjpeg &
PDF Jpeg Support & \\
Pdfops &
Parsing PDF Graphics Streams & \\
Pdfdest &
Destinations & \\
Pdfmarks &
Bookmarks & \\
Pdfpagelabels &
Page Labels & \\
Pdfpage &
Page-level functionality & \\
Pdfannot &
Annotations & \\
Pdffun &
Parsing and Evaluating PDF Functions. & \\
Pdfspace &
Colour Spaces & \\
Pdfimage &
Extract Images. & \\
Pdfafm &
Parse Adobe Font Metrics files & \\
Pdfafmdata &
AFM Data for the standard 14 fonts & \\ 
Pdfglyphlist &
Glyph Lists & \\
Pdftext &
Parsing fonts and extracting text from content streams and PDF strings & \\
Pdfstandard14 &
Standard PDF Fonts & \\
Pdfgraphics &
Structured Graphics. & \\
Pdfshapes &
Basic Shapes & \\
Pdfdate &
Representing and Parsing PDF Dates & \\
Pdfocg &
Optional Content Groups. & \\
Pdfcff &
Convert a CFF Type 1 Font to a Type 3 Font. & \\
Pdftype1 &
Convert an PostScript Type 1 Font to a Type 3 Font. & \\
Pdftruetype &
Convert a TrueType font to a Type 3 Font. & \\
Pdftype0 &
Type 0 font support & \\
Pdfmerge &
Merge PDF files, optionally rotating some pages. 
 \end{tabular}
}

\smallgap
\noindent The HTML documentation for CamlPDF is built in \texttt{doc/html/camlpdf} when \textsf{CamlPDF} is built. You can, of course, eschew the top level and compile projects using the CamlPDF library directly: this gives native speeds and self-contained executables.

\smallgap

\section*{Further Reading}
The author's book is a suitable introduction to the PDF file format:\\
\url{http://shop.oreilly.com/product/0636920021483.do}

\vspace{4mm}

\noindent For any serious work, you will need the PDF Reference Manual\\
\url{http://www.adobe.com/devnet/acrobat/pdfs/PDF32000_2008.pdf}

\vspace{4mm}

\noindent For an introduction to OCaml, the author's book is available:\\
\url{http://ocaml-book.com} or at Amazon.com

\backmatter
\printindex
\end{document}

